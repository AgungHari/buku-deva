\chapter*{ABSTRAK}
\begin{center}
  \large
  \textbf{PENGEMBANGAN SISTEM KENDALI KURSI RODA BERBASIS \emph{SIBI} DAN \emph{BRAKING SYSTEM} MENGGUNAKAN \emph{LSTM} dan \emph{YOLOv11}}
\end{center}
\addcontentsline{toc}{chapter}{ABSTRAK}
% Menyembunyikan nomor halaman
\thispagestyle{empty}

\begin{flushleft}
  \setlength{\tabcolsep}{0pt}
  \bfseries
  \begin{tabular}{ll@{\hspace{6pt}}l}
  Nama Mahasiswa / NRP&:& I Putu Deva Febriana / 5024211016\\
  Departemen&:& Teknik Komputer FTEIC - ITS\\
  Dosen Pembimbing&:& 1. Dr. Supeno Mardi Susiki Nugroho,S.T.,M.T.\\
  & & 2. Dr. Eko Mulyanto Yuniarno,S.T.,M.T.\\
  \end{tabular}
  \vspace{4ex}
\end{flushleft}
\textbf{Abstrak}

% Isi Abstrak
Penelitian ini bertujuan untuk mengembangkan sistem kendali kursi roda berbasis gestur Sistem Bahasa Isyarat Indonesia (SIBI) dan sistem pengereman otomatis (automatic braking system) dengan menggunakan  Long Short-Term Memory (LSTM) dan YOLOv11. Sistem ini dirancang untuk memberikan solusi inovatif bagi pengguna kursi roda, khususnya bagi mereka yang mengalami keterbatasan fisik dan komunikasi, dengan memanfaatkan isyarat tangan SIBI sebagai perintah untuk menggerakkan dan mengendalikan kursi roda. Teknologi YOLOv11 digunakan untuk mendeteksi objek yang berada di depan pengguna kursi roda, sedangkan LSTM diterapkan untuk mengolah urutan gestur dan memprediksi perintah yang diinginkan pengguna menggunakan gestur SIBI. Selain itu, sistem pengereman otomatis dikembangkan untuk meningkatkan keamanan pengguna, dengan memanfaatkan kamera yang dapat mendeteksi rintangan atau situasi darurat. Pengujian menunjukkan bahwa sistem yang diusulkan memiliki tingkat akurasi dan responsivitas yang tinggi, sehingga dapat meningkatkan kemandirian dan mobilitas pengguna kursi roda.

\vspace{2ex}
\noindent
\textbf{Kata Kunci: \emph{Kursi roda cerdas, SIBI gesture, YOLOv11, LSTM, Smart Breaking System.}}